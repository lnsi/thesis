\documentclass[11pt]{amsart}
\usepackage{geometry}                % See geometry.pdf to learn the layout options. There are lots.
\geometry{letterpaper}                   % ... or a4paper or a5paper or ... 
%\geometry{landscape}                % Activate for for rotated page geometry
%\usepackage[parfill]{parskip}    % Activate to begin paragraphs with an empty line rather than an indent
\usepackage{graphicx}
\usepackage{amssymb}
\usepackage{epstopdf}
\DeclareGraphicsRule{.tif}{png}{.png}{`convert #1 `dirname #1`/`basename #1 .tif`.png}

\title{OpenTable - Scenarios}
\author{Leo Sicard}
%\date{}               % Activate to display a given date or no date

\begin{document}
\maketitle

\section{The Coffee Shop}

Alice is supposed to meet her friend Bob in a Coffee Shop.
She arrives early, and chooses to sit at one of the interactive tables.
After ordering a drink via the tabletop, she takes her phone out of her purse (turns on the WIFI) and places it on the table.
A popup dialog window appears next to her smartphone, asking Alice to confirm the establishment of a UI transfer.
Alice accepts, and her phone's screen appears on the larger surface, seemingly attached to her physical device.
She resizes the UI to her convenience, and moves it closer to her by sliding her phone on the surface.
Alice is thus able to use her smartphone's applications on the larger surface while waiting for Bob. She writes and sends an email, and browses the internet.

When Bob arrives, Alice minimizes her screen next to her phone, but keeps the connection active.
Bob orders a drink and they start catching up.
Bob has just returned from a vacation and he has pictures on his smartphone that he wishes to show Alice.
He connects his phone to the table as Alice had done, so that they can both look at the pictures on the larger screen of the table. When done, Bob disconnects his phone by simply lifting it off the table.
After a while, they decide to go watch a movie.
Alice restores her phone's UI on the table and opens a browser in order to access the program of the closest cinema.

\section{The Meeting}

Jim, Jack and Jill are having a meeting about the development of a software product.
They are sitting around an interactive table, with different artefacts, including paper, pens, computing devices and coffee cups.
Jill is directing the meeting, she has her smartphone placed on and connected to the table, its screen extended to a portion of the surface.
On the screen is a text document presenting the agenda, Jack and Jim are also able to easily refer to the different points of the agenda.

Jack's task is to show and explain a diagram to the others.
He switches on his tablet computer, opens said diagram, and places the tablet on the table for the others to see.
Jack connects his device to the surface, thus allowing the tablet's screen to be extended to the surface.
Jack attaches the UI to the table, allowing him to remove the tablet.
He uses simple touch to resize and rotate the window, and presents the diagram to his colleagues.
When done, Jack switches off his tablet computer, which has the effect of interrupting the connection to the table and instantly closing the diagram window.


\section{The Office}

It is monday morning and Bill arrives at his office.
His desk is an interactive table.
On it are a laptop computer, stacks of papers, books, pens, an empty cup and a lamp.
Bill powers on the tabletop and laptop, and places his smartphone on the table.
Bill's smartphone is known to the tabletop, and therefore the UI connection is automatically established in dual view mode, allowing Bill to drag widgets out of his smartphone.
Bill places its calender up in one corner, together with its Skype widget.
After reading through his mail on the laptop computer, Bill starts writing an answer, for which he needs to refer to a document that is stored on the smartphone.
Bill switches to a mirrored display mode for his phone, causing its screen to appear on the table, attached to the device.
By sliding the phone, he moves the display to a convenient location.
Suddenly the phone rings.
Bill attaches all applications and UI display to the table, allowing him to pick up the phone without losing his open document nor widgets.






\end{document}  