\documentclass[11pt]{amsart}
\usepackage{geometry}                % See geometry.pdf to learn the layout options. There are lots.
\geometry{letterpaper}                   % ... or a4paper or a5paper or ... 
%\geometry{landscape}                % Activate for for rotated page geometry
%\usepackage[parfill]{parskip}    % Activate to begin paragraphs with an empty line rather than an indent
\usepackage{graphicx}
\usepackage{amssymb}
\usepackage{epstopdf}
\DeclareGraphicsRule{.tif}{png}{.png}{`convert #1 `dirname #1`/`basename #1 .tif`.png}

\title{TIDE user experiment}
\author{lnsi}
%\date{February 2012}                                           % Activate to display a given date or no date

\begin{document}
\maketitle

\emph{Introduction speech to the user:}
\\\\
``The experiment is divided in 3 parts.
First, you will learn how to use the application.
Second, I will guide you through two scenarios where you will use the application to do specific things.
Third, I will ask you to answer to a questionnaire.

The Microsoft Surface is a tabletop computer with an interactive display.
You interact with it by using finger touch, like on a smartphone.

The TIDE application allows you to connect a smartphone to the tabletop.
To do that, you only need to put the phone on the table.
An application window appears on the table screen, which resembles the phone.
You can manipulate the window, but if you touch inside, the input will be relayed to the phone.

This is a prototype, implying that the application presents some defaults.
The main issue is that there is an important lag when interacting with the phone, please be patient.
Unexpected behavior might occur, in which case I will step in to keep the experiment on track.''

\section{Discovering Tide (10 min)}

The user discovers the application and learns to use its features.
First by himself, then with the support of the designer.

\subsection{Free exploration}
 
The user has 5 minutes to freely try the prototype, understand how it works and discover its features.

\subsection{Features}

User and designer go through each command together.
For each command, the user shows which actions he knows.
If there are any unknown actions, the designer teaches them to the user.

The designer fills out the form Tide-EF1.

\section{Using Tide (10 min)}

The designer guides the user through two scenarios where the user performs various tasks using the application.
The designer tells the user what to do, but not how to do it.

\subsection{Gaming}

Scenario: user and designer are two friends in a waiting room. They play a game of tic-tac-toe to pass the time.

\begin{enumerate}
\item connect the iPhone to the tabletop
\item launch the game called 'tic tac toe'
\item start a 2 players game
\item position and resize the window to allow for 2 players to interact
\item now we play
\item exit the game
\item close the application
\end{enumerate}

\subsection{Browsing}

Scenario: user and designer are planning a surprise dinner at a restaurant for the birthday of a friend. The user uses the application to go online and find the address of a restaurant, then to find how to go from the university to the restaurant.
When the friend is close, the user should hide what he is doing.

\begin{enumerate}
\item connect the HTC Legend to the tabletop
\item adjust the position and size of the window
\item launch the app called Internet
\item find the address of 'restaurant aristo'
\item minimize the window to hide what you are doing
\item restore the window
\item launch the app called Maps
\item look up the address for the restaurant
\item zoom in on the restaurant
\item find out how to walk from ITU to the restaurant
\item rotate the window to show me how to walk from ITU to the library
\item minimize the window to hide what you are doing
\item restore the window
\item return to phone home screen
\item exit application
\end{enumerate}

\section{Questionnaire (10 min)}

The user fills in the evaluation form Tide-EF2.

%teaching phase for each primitive (5):
%1) ask them to do it
%	- do they know how to do it
%	- which technique do they use?
%2) ask them if they know of other technique
%- if yes, which? show?
%- if no, tell them. show?
%
%scenario phase:
%1) browsing:
%- pair Legend
%- look up sthg online: remote UI, dragging, resizing
%- shoe me sthg: rotating
%- hide stag from someone: minimizing/hiding
%
%2) gaming (designer is opponent)
%- pair iPhone
%- launch game: remote UI
%- resize to fullscreen (maximize)
%- exit
%
%questionnaire phase: google forms?
%= straightforward user satisfaction study
%- rate the experience: usable? engaging?
%- would you use such a system?
%- how intuitive is the system?
%
%
%LOGGING (user, session, command, action, timestamp)
%
%drag, resize, rotate = using active border
%
%maximize by size enlarge
%rotate when maximized
%
%minimize by drag to edge
%minimize by double tap
%minimize by blob touch
%minimize by size reduce
%
%close by minimize
%close by corner
%close by press and hold

\end{document}