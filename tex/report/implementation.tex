
%%%%%%%%%%%%%%%%%%%%%%
%%% IMPLEMENTATION %%%
%%%%%%%%%%%%%%%%%%%%%%

refer to problem formulation:

Following is an open list of problems that we will address in order to achieve device composition by means of implicit interaction.
\begin{enumerate}
\item{\emph{Setup}: How is a device enabled for integrating with a tabletop?
The setup should be simple, to be performed only once by non-technical users.
An initial survey of possible solutions points towards the use of tagging mechanisms and/or camera-based object recognition.}
\item{\emph{Discovery}: How do the tabletop and the device discover and communicate with each other?
How do we solve the issues of discovery, handshake, network connectivity, and encryption mechanisms to ensure privacy?}
\item{\emph{UI transfer}: Given the computational constraints of mobile devices, how can the UI transfer be efficiently implemented so as to support native applications and guarantee a seamless user experience?}
\item{\emph{Input}: How can the users interact with their applications on the tabletop (touch and other peripherals)?}
\item{\emph{Interaction Design}: What means of interaction are best-fitted for the tabletop-based systems that we propose to develop?
How can we best adapt to public/private uses and single/multiple users?
How can we take advantage of the larger interaction surface?}
\end{enumerate}

\hfill\\

1) setup

2) discovery

--> how it is not new, what are the existing options, what would I recommend in this context. Discussion. How did I solve it and why.

3) vision-based device tracking
detection options: iPhone App, Tag, camera based 

4) UI transfer (I/O approach)
- technology issues (slow Veency)

5) Interaction design

6) generic implementation attempt 
include android phone