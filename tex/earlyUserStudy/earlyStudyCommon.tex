\documentclass[11pt]{amsart}
\usepackage{geometry}                % See geometry.pdf to learn the layout options. There are lots.
\geometry{letterpaper}                   % ... or a4paper or a5paper or ... 
%\geometry{landscape}                % Activate for for rotated page geometry
%\usepackage[parfill]{parskip}    % Activate to begin paragraphs with an empty line rather than an indent
\usepackage{graphicx}
\usepackage{amssymb}
\usepackage{epstopdf}
\DeclareGraphicsRule{.tif}{png}{.png}{`convert #1 `dirname #1`/`basename #1 .tif`.png}

\title{Displex early user study}
\author{lnsi}
%\date{}                                           % Activate to display a given date or no date

\begin{document}
\maketitle

\section{Introduction}
``You are here to participate in a short usability study whose purpose is to assist in the design of a software application called Displex.
The experiment will last about 30 minutes and is being recorded, both as audio and video.
I will give you instructions by reading from this script, so as to give you the same information as to the other participants.
After this introduction you will be able to ask questions before we start the experiment.

First, let me introduce the application.
Displex is a software application that connects a smartphone (in this case an iPhone) with a Microsoft Surface tabletop computer.
It allows you to interact with your smartphone on a larger screen, by transfering the display of your smartphone to the screen of the interactive surface.
Do you understand the basic concept of the application?

During this experiment, I will ask you to perform a task using the application.
This will lead you to perform a number of actions that use the basic features of Displex.
For each action, there will be 3 steps:

\begin{itemize}
\item{First, I will explain the action, and show you its effect on this screen}
\item{Second, you will describe to me how you would suggest performing this action with the user interface of Displex.}
\item{Third, I will describe 3 different ways of performing the action, and ask you to order the suggestions by order of your preference.}
\end{itemize}

This experiment is based on prototypes, meaning that we will use the available paper representations in order to describe the user interface of Displex.
There are paper, pen and scissors available for building your own prototypes if necessary.
We will also use the iPhone, the MS Surface, and of course words.

Do you have any questions concerning the general course of the experiment?
\\\\
Let us begin.
Your general task is to write an email to a friend using your iPhone and the Displex application on the Microsoft Surface.
We will talk about 7 basic actions.''

\pagebreak
