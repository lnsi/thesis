\documentclass[11pt]{amsart}
\usepackage{geometry}                % See geometry.pdf to learn the layout options. There are lots.
\geometry{letterpaper}                   % ... or a4paper or a5paper or ... 
%\geometry{landscape}                % Activate for for rotated page geometry
%\usepackage[parfill]{parskip}    % Activate to begin paragraphs with an empty line rather than an indent
\usepackage{graphicx}
\usepackage{amssymb}
\usepackage{epstopdf}
\DeclareGraphicsRule{.tif}{png}{.png}{`convert #1 `dirname #1`/`basename #1 .tif`.png}

\title{Displex early user study}
\author{lnsi}
%\date{}                                           % Activate to display a given date or no date

\begin{document}
\maketitle

\section{Introduction}
``You are here to participate in a short user study whose purpose is to help in the development of a software application called Displex.
The experiment will last 20 minutes and is being recorded.
I will give you instructions by reading from this document only.
This way, you will receive precisely the same information as the other participants.
I will of course answer any further questions you might have.

First, let me introduce the application.
Displex is a software application that connects a smartphone (in this case an iPhone) with a Microsoft Surface tabletop computer.
It allows you to interact with your smartphone on a larger screen, by transfering the display of your smartphone to the screen of the interactive surface.
Do you have any questions concerning the basic concept of the application?

During this experiment, I will present you with several ground features of the application.
Those features represent the basic actions that a user would perform when using Displex.
For each of those actions, I will guide you through the following steps:

\begin{itemize}
\item{I will explain the action, and show you its effect on this screen}
\item{I will ask you to describe how you would imagine performing this action with the User Interface of Displex.
To this purpose, you are welcome to use words, and you can use the Microsoft Surface and iPhone, as well as all the available paper prototypes. If necessary, you can use paper, pen and scissors to create a prototype of your choice.}
\item{Finally, I will describe 3 different ways of performing the action, and ask you to choose the one you prefer.}
\end{itemize}
Do you have any questions concerning the course of the experiment?''

\pagebreak

\section{Experiment}

\subsection{Pairing}
\hfill\\
In order to be able to access your iPhone on the Microsoft Surface, you connect the phone to the tabletop as well as launch the Displex application.
\\\\
Concerns: security (dialog on phone) \& obtrusiveness (dialog on phone first)
\\\\
Suggestions:
\begin{description}
\item[A]{The application launches automatically when the smartphone is placed on the surface, and a dialog window appears on the smartphone, offering the user to establish the connection.}
\item[B]{The application launches automatically when the smartphone is placed on the surface, and 2 dialog windows appear, first on the surface, then on the smartphone, offering the user to establish the connection.}
\item[C]{The application launches automatically when the smartphone is close enough to the surface, and a dialog window appears on the surface, offering the user to establish the connection.}
\end{description}

\subsection{Dragging}
\hfill\\
Once your iPhone screen is transferred and active on the tabletop, you drag the window closer to yourself in order to interact with it more conveniently. You open your email application.
\\\\
Concerns: inner window is active (input to phone)
\\\\
Suggestions:
\begin{description}
\item[A]{The user performs a one finger dragging gesture on a specific tab.}
\item[B]{The user performs a one finger dragging gesture on the active border of the window.}
\item[C]{The user taps a tab to render the window inactive, then performs a one finger dragging gesture anywhere on the window.}
\end{description}

\subsection{Rotating}
\hfill\\
The orThis action has the effect of rotating the application window. 
\\\\
Suggestions:
\begin{description}
\item[A]{The user performs a two finger touch rotating gesture with one finger placed on a specific tab, and the other finger anywhere on the window.}
\item[B]{The user performs a two finger touch rotating gesture on a large tab.}
\item[C]{The user taps a tab to render the window inactive, then performs a two finger touch rotating gesture anywhere on the window.}
\end{description}

\subsection{Resizing}
\hfill\\
This action has the effect of enlarging or reducing the application window.
\\\\
Suggestions:
\begin{description}
\item[A]{The user performs a one finger dragging gesture on one of the active corners of the window.}
\item[B]{The user performs a one finger dragging gesture on a specific tab.}
\item[C]{The user taps a tab to render the window inactive, then performs a two finger pinching gesture anywhere on the window.}
\end{description}

\subsection{Minimizing}
\hfill\\
This action has the effect of reducing the application window to an icon, keeping the application alive while moving it to the periphery.
\\\\
Suggestions:
\begin{description}
\item[A]{The user taps a specific tab.}
\item[B]{The user performs a one finger dragging gesture on one of the active corners of the window, reducing it until it becomes an icon.}
\item[C]{The user double taps one of the active corners of the window.}
\end{description}

\subsection{Restoring}
\hfill\\
This action can only be performed when the application is minimized, it has the effect of restoring the window to its previous size.  
\\\\
Suggestions:
\begin{description}
\item[A]{The user taps a specific tab.}
\item[B]{The user taps the icon.}
\item[C]{The user double taps the icon.}
\end{description}

\subsection{Hiding}
\hfill\\
It is the action of 
\\\\
Suggestions:
\begin{description}
\item[A]
\item[B]
\item[C]
\end{description}

\subsection{Exiting}
\hfill\\
It is the action of 
\\\\
Suggestions:
\begin{description}
\item[A]
\item[B]
\item[C]
\end{description}

\end{document}  
